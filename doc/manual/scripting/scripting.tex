% vim: ts=2:sw=2:tw=80:et
\thispagestyle{fancy}
\pagestyle{fancy}

\section{Global Script}
The \textbf{Global Script} facility allows the user to fully define and
customize the Python environment within which Arbwave runs.  For instance,
modules can be imported to extend the computational tools available to the user.
Furthermore, functions and classes can be defined to make the users workflow
more streamlined or simple.  This script is generally executed when a
configuration file is loaded or any other change to the \textbf{Global Script}
is made.  It is also possible for the user to manually request the runtime
environment to be reinitialized and the \textbf{Global Script} to be executed
anew.  This is done via the command-prompt interface using the \texttt{reset()}
function.

Using the \textbf{Global Script}, one can initialize connections to cameras,
initialize serial (USB) hardware, or perform any number of other operations.

\subsection{Arbwave Command Line}
One way to immediately modify or test the environment established by the
\textbf{Global Script} is to use the \textbf{Arbwave Command Line} interface
provided on the main window.  This interface is a Python command interpreter
operating within the confines of the global environment.  This means that any
variables or expressions executed in the \textbf{Arbwave Command Line}
immediately effect the global environment.

\subsection{Variables} \label{script:variables}
If you want to use a variable in the waveform element lines, you should
probably define that variable in the \textbf{Global Script}.  By
doing so, these variables will be available everywhere and also at any time.
You can then modify these variables by specifying them as loop variables in
the loop editor and selecting ``Global''.

Any valid python expression that can be assigned can be used as a loop
variable.  For instance, consider the following variables initialized in the
global script:

\begin{lstlisting}
  Bfield = [0*Gauss, 10*Gauss, 123*Gauss]
  CoilCurrent = 4.2*A
  Frequency = dict(mot = -10*MHz, pgc = -40*MHz, image=0*MHz)
\end{lstlisting}

These expressions and any valid Python expression involving these variables
can be used anywhere in Arbwave.


\subsection{Arbwave Python Modules}
Arbwave provides two modules that can be imported for extending or probing the
functionality in Arbwave.  The first, \textbf{Arbwave}, is a virtual module with
direct connections to the running process engine.  The second, \textbf{arbwave}
(all lower-case), is the module hierarchy of the Arbwave code implmentation.
This module is generally not for everyday use, but rather for exploring and
testing some of the internals of Arbwave.

\subsubsection{\textbf{Arbwave} Module (virtual)}
The \textit{Arbwave} module is a virtual module with access to the running
process Arbwave engine.  Among other things, with the functions of this virtual
module, it is possible to force a hardware update, save the waveform to an
external output file.  The available functions from this virtual module are as
follows.

\begin{lstlisting}
  Arbwave.add_runnable(label, runnable):
    """
    Adds a Runnable class to the list of possible runs.
    """

  Arbwave.update(stop=ANYTIME, continuous=False, wait=True):
    """
    Process inputs to generate waveform output and send to
    plotter.
      <b>stop</b>    [Default : Arbwave.ANYTIME]
        The time at which to test for stop requests during the
        update function.
        Valid values:
          None
            Do not check for stop requests during the update
            function.
          Arbwave.BEFORE
            Before the waveforms are computed and sent to
            hardware.
          Arbwave.AFTER
            After the waveforms are computed and sent to
            hardware.
          Arbwave.ANYTIME
            Either before or after waveforms are computed
            and sent to hardware.
      <b>wait</b>
        Whether to wait until the non-continuous waveform has
        completed before returning from the update function.
    """

  Arbwave.update_static():
    """
    Only process static outputs and send to plotter.
    """

  Arbwave.update_plotter():
    """
    Process inputs to send only to plotter.
    """

  Arbwave.stop_check():
    """
    Check to see if a request to stop has been issued and
    stop if so.
    """

  Arbwave.save_waveform(filename, fmt='gnuplot'):
    """
    Save waveform to a file in the specified output format.
      filename : the file to save the waveform to
        If the output file format is for gnuplot, then the filename can be:
          (1) a filehandle of an open file to write to
          (2) a name of a file to open then write to
          (3) or filename format to use for creating files specific to each
              clock.
              Example:  file-{}.txt
               {} will be replaced with the clock name
          If a single filename is specified, everything is saved in the same file

      fmt : either 'gnuplot' or 'python' to specify the output format
    """

  Arbwave.find(label, index=0):
    """
    Search for either channels or groups at the current level.
      label : name of channel or group
      index : select the ith occurance of a channel/group label
    """

  Arbwave.find_group(label, index=0):
    """
    Search for group at the current level.
      label : name of group
      index : select the ith occurance of a group label
    """

  Arbwave.find_channel(label, index=0):
    """
    Search for channel at the current level.
      label : name of channel
      index : select the ith occurance of a channel label
    """
\end{lstlisting}

\subsubsection{\textbf{arbwave} Module}
The \textbf{arbwave} (all lower-case) module is the module hierarchy of the
Arbwave code implmentation.  This module is generally not for everyday use, but
rather for exploring and testing some of the internals of Arbwave.  This module
is particularly useful in debugging the development of new backend drivers.


\subsection{Experimental Control Loop}\label{sec:scripting:runnable}
\subsubsection{Basic Customization}\label{sec:scripting:simple}

\begin{lstlisting}
  class SimpleRun(Arbwave.Runnable):
  	def run(self):
  		Arbwave.update()
  		return 0
\end{lstlisting}


\begin{lstlisting}
  Arbwave.add_runnable( 'Simple', SimpleRun() )
\end{lstlisting}

\subsection{Advanced Customization}

\begin{lstlisting}
  class Runnable(object):
    """
    Runnable class to allow more significant complexity during a data run.
  
    This implementation mostly defines the interface that runnables provide, but
    this is also the implementation of the "Default" runnable that is used to
    execute output in continuous mode.
    """
    def extra_data_labels(self):
      """
      Returns list of names of extra results returned by self.run()
      """
      return list()
  
    def onstart(self):
      """
      Executed before the runnable is started.
      """
      pass
  
    def onstop(self):
      """
      Executed after the runnable is stopped.
      """
      pass
  
    def run(self):
      """
      The body of any inner loop.
      """
  	def run(self):
  		Arbwave.update()
  		return 0
\end{lstlisting}

\section{Local Scripts}
In addition to executing scripts that affect the entire experiment globally, it
is also possible to exucute scripts that have as narrow a scope as a group.  In
order to specify a local script, the group menu is used to select \textbf{Local
Script}.  Saving the local script will cause it to be immediately executed as
the entire waveform is updated.  Each additional waveform update will cause a
local script to execute.
