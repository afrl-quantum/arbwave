% vim: ts=2:sw=2:tw=80:et
\thispagestyle{fancy}
\pagestyle{fancy}

\section{Version Numbers}
Versions are generally defined using tags of the \acro{Git} revision control
system.  Generally, all Arbwave tags (and hence version strings) should follow
the format
\begin{verbatim}
    arbwave-x.y.z
\end{verbatim}
  where:
\begin{itemize}[leftmargin=2cm]
  \item[\textbf{x :}] Major revision with major differences of capabilities as compared to
      other major revisions.  This definition of major differences is somewhat
      subjective and will be mostly at the whim of maintainers.
  \item[\textbf{y :}] Minor revision with incompatible differences of interfaces as compared
      to other earlier revisions.  Interfaces that are considered to impact the
      minor revision number are external interfaces such as the Pyro remote
      object interfaces or changes in the standard programming interfaces
      provided by the `Arbwave` and `arbwave` modules.
  \item[\textbf{z :}] Micro revision indicating a general acceptance of multiple patches since
      last tag.  This number may be used to help mark minor development
      milestones.
\end{itemize}
%
By using the Git command `git describe`, a unique identifier of the full
version string can be shown as
\begin{verbatim}
    arbwave-x.y.z[-n-g<sha1>]
\end{verbatim}
  where \verb|[-n-g<sha1>]| shows up automatically if changes have been made
  since the last tag and
\begin{itemize}[leftmargin=2cm]
  \item[\textbf{n :}] Indicates the number of commits since the last tag
  \item[\textbf{g\textless sha1\textgreater :}] Indicates the abreviated SHA1
    hash of the latest commit
\end{itemize}

\vspace{1em}
For Arbwave installations to be used where \acro{Git} is not available, a static
file containing the version can be deployed.  This file can be manually created
from a machine with \acro{Git} installed by executing the following command
\begin{verbatim}
    python python/arbwave/version.py save
\end{verbatim}


\section{Save-File Format}
Arbwave uses a native Python file as its configuration storage format.  The
format is defined by the structure of the data members that are kept in the
configuration file.  Because Python is used, it is possible to load the
configuration file into a Python shell for examination.  Furthermore, if
desired, it is possible for the save-file to (at least until it is overwritten)
include various complex Python logic that modifies the configuration structures.

The definitions of the data members are tied to the Arbwave version number with
various rules:
\begin{itemize}
  \item Changes in the save-file format may occur for any change in x,y, or z,
        but \textbf{will} occur only with at least one of x,y, or z changing.
  \item The Save-File format includes a marker of the Arbwave version that was
        used to save the file.
  \item Arbwave uses the version information to automatically upgrade
        experimental configuration files to the newest format.
\end{itemize}
