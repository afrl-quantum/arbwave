% vim: ts=2:sw=2:tw=80:et
\thispagestyle{fancy}
\pagestyle{fancy}

This manual describes the use and operation of Arbwave, the arbitrary waveform
experimental control program.

\section{What is Arbwave?\index{What is Arbwave?}}
\subsection{Background}
Atomic physics experiments typically involve various voltages, currents,
magnetic fields, laser fields, and mechanical devices that must be manipulated
and altered according to very precise relative timing relationships.  It is
necessary to use hardware and software that provides for coordinating these
tight timing relationships.  The extent to which software and hardware are
able to present these timing relationships to a user in a succinct and
consistent fashion directly determines how well a configuration can be
tailored to the specific experimental situation.

\subsection{What can Arbwave do?}
Arbwave provides a means for a user to clearly coordinate groups of transitions
for various control signals that can be used to run a complicated experimental
timing sequence.  In Arbwave, a series of channels are defined where each
channel represents some control signal used to supervise an experimental
parameter such as voltage, current, magnetic field, etc.  Using the
information of the defined channels, Arbwave allows one to coordinate
heirarchical sets of transitions for each of those channels.  Arbwave also
graphically represents each of the signal transitions as complete waveforms
where the time for each transition is accurately shown.  Furthermore, Arbwave
facilitates execution of these waveforms in various operation patterns, such
as control-variable nested loop iteration and multi-variable optimization.
Finally, user scripting, via Python, can be inserted into the execution to
customize a particular experimental procedure, customize an optimization
procedure, and hook into various external data recorders processors.


\section{Supported Hardware\index{Supported Hardware}}\label{sec:hardware}
\begin{center}
\tablefirsthead{%
  \hline
  \textbf{\large Manufacturer} &
  \textbf{\large Model} &
  \textbf{\large Description} &
  \textbf{\large Supported Driver} \\
  \hline
  \hline
}
\tablehead{%
  \hline
  \multicolumn{4}{|l|}{\small\sl continued from previous page}\\
  \hline
  \textbf{\large Manufacturer} &
  \textbf{\large Model} &
  \textbf{\large Description} &
  \textbf{\large Supported Driver} \\
  \hline
  \hline
}
\tabletail{%
  \hline
  \multicolumn{4}{|r|}{\small\sl continued on next page}\\
  \hline
}
\tablelasttail{\hline}
\bottomcaption{Supported Hardware}
%
%
\begin{supertabular}{|l|c|l|r|}
  NI & PCI-6221 & 16-bit Analog output, n-channel & NiDAQmx \\
     &          &                                 & comedi (in progress) \\
  NI & PCI-6225 & 16-bit Analog output, n-channel & NiDAQmx \\
     &          &                                 & comedi (in progress) \\
  NI & PCI-6229 & 16-bit Analog output, n-channel & NiDAQmx \\
     &          &                                 & comedi (in progress) \\
  NI & PCI-6723 & 12-bit Analog output, 32-channel & NiDAQmx \\
     &          &                                 & comedi (in progress) \\
  NI & PCI-6733 & 16-bit Analog output, 8-channel & NiDAQmx \\
     &          &                                 & comedi (in progress) \\
  NI & PXI-6733 & 16-bit Analog output, 8-channel & NiDAQmx \\
     &          &                                 & comedi (in progress) \\
  Viewpoint & DIO-64 & Digital input/output, 64-channel & DIO64 \\
  Marvin Test & GX164x & 16-bit Analog output, 64-channel & GxAo (in progress) \\
  Marvin Test & GX3500 & FPGA based digital output, 120-channel & GxFpga \\
\end{supertabular}
\end{center}
