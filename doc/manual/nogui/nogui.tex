% vim: ts=2:sw=2:tw=80:et
\thispagestyle{fancy}
\pagestyle{fancy}

For some cases, it can be very helpful to control experiments and acquire data
without the need to have a graphical user interface.  There are several
situations where a non-gui interface may be preferable.  For example, it may be
desireable to use an embedded system with limited graphical interface options to
provide experiment command and control and data acquisition.  It may also be
necessary to execute waveforms in an environment without user interactions.
Using a special interface to the Arbwave computational engine, and backend
drivers, it is possible to load Arbwave configurations built with the gui and
run them in the same manner that one does through the graphical user interface.

\section{Quick Start}

The following serves to demonstrate a very simple method of loading one of the
Arbwave examples in simulated mode using the non-graphical interface.

\begin{lstlisting}
from arbwave.nogui import Arbwave

# Start the Arbwave interface in simulated mode
arbw = Arbwave(simulated=True)
arbw.open('examples/simple_config.py')

# start/stop the waveform in Hardware-repeated mode:
arbw.start()
arbw.stop()

# run the waveform using the Simple runnable:
arbw.active_runnable = 'Simple'

# run 'Simple' only 'once' executor
arbw.start()

# run 'Simple' with the loop executor and get the data from the loop
arbw.executor = 'loop'
arbw.start()

# Get the data as a numpy array
data = arbw.datalog.tables[0]
\end{lstlisting}

\section{Non Graphical User Interface}

\input{arbwave.nogui}
