% vim: ts=2:sw=2:tw=80:et
\thispagestyle{fancy}
\pagestyle{fancy}

\section{Prerequisites\index{Installing prerequisites}}

This project depends on several other python-related projects and these must
first be installed.

\subsection{Dependencies}
Arbwave is dependent on various packages, both generally available and also
several that are distributed with the arbwave project repository.  This list
enumerate the various specific packages that Arbwave depends on.  Those
distributed as a part of the Arbwave project repository are marked below.

\begin{itemize}
  \item Python packages/libraries:
    \begin{itemize}
      \item \textbf{python (2.7)} \\
        Python version 3.X will not currently work.
      \item \textbf{pygi} \\
        \textbf{pygi} is the base package of the GObject instropsection library
        interface.  This library provides dynamic access to various other
        libraries, including the Gtk runtime.  In addition to installing
        \textbf{pygi}, one must also have these installed (often already installed
        on Gnome-based Linux systems):
        \begin{itemize}
          \item Gtk $> 3.10$
          \item Pango $> 3.10$
        \end{itemize}

      \item \textbf{matplotlib} \\
        This package also provides mpl\_toolkits import.
      \item \textbf{numpy} \\
        Generally, this is included as part of scipy and required for matplotlib.
      \item \textbf{scipy} \\
        scipy.interpolate imports are required for channel scaling/calibration.
      \item A graph library for graph-analysis of connections between hardware
        components.\\
        You must have either of these installed:
        \begin{itemize}
          \item \textbf{python-igraph} (\underline{Significantly} faster)
          \item \textbf{networkx} (Apparently more available)
        \end{itemize}

      \item \textbf{Pyro}; version 3, not 4

      \item \textbf{pylibnidaqmx}
        \begin{itemize}
          \item must have modifications by S Olson
          \item Python wrapper library (wraps NI DAQmx nicaiu.dll c library)
            provides actual linkage to the NIDAQmx drivers
          \item Only needed if NIDAQmx relevant hardware is going to be used
        \end{itemize}

      \item \textbf{physical}
        \begin{itemize}
          \item http://github.com/olsonse/physical
          \item dimensional analysis
          \item units and constants library (c++, python, octave, gnuplot)
        \end{itemize}

      \item \textbf{viewpoint}
        \begin{itemize}
          \item python wrapper library (wraps Viewpoint dio64\_32.dll c library)
          \item only required if using ViewpointUSA dio64 hardware
        \end{itemize}

      \item \textbf{marvin}
        \begin{itemize}
          \item python wrapper library around MarvinTest c-libraries
          \item only required if using MarvinTest FPGA or analog output hardware
        \end{itemize}

      \item \textbf{billiard}
        \begin{itemize}
          \item must have modifications by S Olson
          \item Not required for Windows platforms
          \item backport of python3 multiprocessing library
        \end{itemize}
    \end{itemize}

  \item C libraries:
    \begin{itemize}
    \item \textbf{NI DAQmx (nicaiu.dll)} \\
      Only needed if NIDAQmx relevant hardware is going to be used.
    \item \textbf{Viewpoint DIO64 (dio64\_32.dll)} \\
      Only needed if Viewpoint's DIO64 digital output/input board will be used.
    \item \textbf{Marvin GxFpga (libGxFpga.so, or GxFpga.dll)}
      Only needed if MarvinTest GxFpga with \acro{AFRL} Timing-Generator
      firmware is used.
    \item \textbf{Marvin GxAo (libGxAo.so, or GxAo.dll)}
      Only needed if MarvinTest GxAo analog output hardware is used.
    \end{itemize}
\end{itemize}

\section{Installation\index{Installing Arbwave}}
This is not necessarily a strict set of instructions for ``installing'' Arbwave,
but rather a set of guidelines that might be used to get Arbwave operational.

\subsection{Windows Platforms}

Note that you may need to specify a web proxy:\\
to get conda to use our web proxy, I had to do:
\begin{verbatim}
set http_proxy=http://proxy.server:port
set https_proxy=http://proxy.server:port
\end{verbatim}


These are the steps I took to get Arbwave functional:
\begin{enumerate}
  \item Install Anaconda python (2.7)
  \item install \verb|pygi-aio-3.x.x_ref<N>-setup.exe| \\
    This is available from pygnome project.
    During the installation process, make at least the following choices:
    \begin{itemize}
      \item GNOME libraries:
        \begin{itemize}
          \item Base packages
          \item GTK+ 3.x.x
          \item Pango 1.x.x
        \end{itemize}
      \item non-GNOME libraries:
        \begin{itemize}
          \item none
        \end{itemize}
      \item development packages:
        \begin{itemize}
          \item none
        \end{itemize}
    \end{itemize}

  \item Install Pyro (v3) \\
    I simply used Anaconda to install the latest version of Pyro.  Doing a
    search, like, \verb|anaconda search -t conda pyro| yielded a source such
    that I was able to install Pyro by doing:
    \begin{verbatim}
      conda install -c david_baddeley pyro
    \end{verbatim}

  \item Install graph library [you can skip this step] \\
    It seems that \textbf{networkx} is already installed by default with
    anaconda and I could not figure out how to install \textbf{python-igraph}
    cleanly.

  \item Do either of these:
    \begin{enumerate}
      \item  (start cmd.exe)
        \begin{enumerate}
          \item cd path/to/arbwave-project
          \item Execute 'python project-shell'
          \item cd arbwave/
          \item 'python run.py [options]'\\
            use --help to see options
        \end{enumerate}
      \item
        \begin{enumerate}
          \item Put all the other packages in your PYTHONPATH somehow
          \item cd path/to/arbwave-project/arbwave/
          \item 'python run.py [options]'\\
            use --help to see options
        \end{enumerate}
    \end{enumerate}
\end{enumerate}



\subsection{Ubuntu}
Ubuntu makes this very easy.  To install most of the non-project components, one
simply does:

\begin{verbatim}
apt-get install \
  python-matplotlib python-scipy python-igraph pyro \
  py-gi libgtk-3-dev libpango1.0-dev
\end{verbatim}


Then, very similar to the last part of the Windows guidelines: \\
Do either of:
\begin{enumerate}
  \item
    \begin{enumerate}
      \item cd path/to/arbwave-project
      \item Execute project-shell
      \item cd arbwave/
      \item ./run.py [options]\\
        use --help to see options
    \end{enumerate}
  \item
    \begin{enumerate}
      \item Put all the other packages in your PYTHONPATH somehow
      \item cd path/to/arbwave-project/arbwave/
      \item ./run.py [options]\\
        use --help to see options
    \end{enumerate}
\end{enumerate}
