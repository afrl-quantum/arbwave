% vim: ts=2:sw=2:tw=80:et
\thispagestyle{fancy}
\pagestyle{fancy}
\section{Loops}
  \subsection{Variables}

  You will find in multiple places that the "insert" key does special things
  like inserting another loop variable.  You can sibling loops, or nested loops.
  You can reorder these by dragging the loop variables.  You can also nest loops
  by dragging/dropping loop variables on top of each other.  Furthermore, while
  dragging, if you press the Control key, the item being dragged will be copied
  instead of moved.  This is also a common interface for editing waveform
  elements.
  Feature request could be:  (1) add some documentation :-) or (2) add a tooltip
  that says to use the insert key to insert a new variable. 



  The insert key is also helpful if you want to define multiple waveforms (when
  you push the button below the "delete waveform element" button or select from
  the waveform popup menu).


  \subsubsection{Using Global Variables}
  You can use any of the variables defined like in Sec.~\ref{script:variables}
  in appropriate waveform elements or even any other place you wish.  In fact,
  you can also use these expressions as loop variables if you select "Global" in
  the loop editor.  As some examples:

  \begin{table}[ht!]
    \center
    \begin{tabular}{p{5cm} m{6cm}c}
    Variable                  &                iterable            &   Global\\
    \hline \hline
    \verb|Bfield[0]|          &       \verb|r_[0:5:3j] * Gauss|      &    X \\
    \verb|Bfield[1]|          &       \verb|r_[5:15:5j] * Gauss|     &    X \\
    \verb|Bfield[2]|          &       \verb|r_[110:130:10j] * Gauss| &    X \\
    \verb|CoilCurrent|        &       \verb|r_[4:5:20j] * A|         &    X \\
    \verb|Frequency['mot']|   &       \verb|r_[-5:-20:5j] * MHz|   &      X \\
    \verb|Frequency['pgc']|   &       \verb|r_[-20:-80:10j] * MHz| &      X \\
    \verb|Frequency['image']| &       \verb|r_[-5:5:10j] * MHz|    &      X \\
    \end{tabular}

    \caption{
      Example of some valid loop variables.
    }
  \end{table}




\section{Optimization}
